\documentclass[acmtocl]{acmtrans2m}

\title{Visualising Courtship Index from Videos of Drosophila}

\author{Michael Dewar, Tim Lukins and Douglas Armstrong}
\begin{abstract}
We present approach to the automated analysis and visualisation of courtship index (CI) in \emph{Drosophila} - usually defined as the percentage of time the male fruit fly spends displaying courtship behaviour over a fixed period. Using machine vision techniques we extract position and orientation features from video of courting flies. These features are then used in a classifier in order to determine the presence of courtship behaviour in each frame. Using this information we generate a set of visualisations which emphasises the time course of the behaviour. Hence, we arrive at a much richer interpretation of courtship index, one that highlights the dynamic aspects of \emph{Drosophila} courtship behaviour, exposing phenotypic information that the standard, manual assay would typically miss.
\end{abstract}

\category{H5.m}{Information interfaces and presentation}{Misc}

\begin{document}
	
	\maketitle

\section{Introduction}

Courtship index (CI) in \emph{Drosophila} is defined as the percentage of time spent by the male exhibiting any kind of courtship behaviour in a given period \cite{}. It is typically collected manually, by watching videos of courting flies and starting and stopping a stopwatch when courtship behaviour is observed. CI is a useful summary statistic, used to quantify a phenotypic change, and to correlate against a change in neurological structure \cite{}.

A trained individual is capable of recording courtship in four flies at a time, or in one fly at up to 4$\times$ real-time video playback. This method of recording courtship index involves two gross forms of information attenuation. 
The first is the attenuation of phenotypic information - compressing a large number of different behaviours into a single ``courtship'' behaviour (such as singing, chasing, dancing etc). 
The second is the attenuation of temporal information - compressing the time course of courtship behaviour into a single percentage value. 
This paper aims to address the second of these issues, using automatic machine vision and machine learning techniques to overcome the inherent limitations of manual analysis that necessitates temporal attenuation.

The use of automated methods of analysis to overcome this attenuation has a set of additional benefits. 
The most striking of these benefits in practice is the reduction of labour - the methods reported below work faster than real time, removing the need to perform routine viewing of the collected videos and allowing the biologist to focus on analysing novel phenotypes that arise. 
Another benefit is consistency - the methods reported below will analyse the same video in the same way every time. This allows the comparisons of results across studies without dealing with (or guiltily ignoring) the problem that different individuals can analyse the same video quite differently. 
The final benefit is the ability to publish - the methods reported below will allow the publication of the classifier used to generate the results along with the collected video and feature data. 

Our approach has two main components: tracking and classification. 
The tracking component combines a background model with a sequence of image manipulations in order to extract ellipses around both male and female flies. 
The classification component uses a labelled set of videos along with their associated features to train a decision tree, which can be used to classify any subsequent videos. 
The remainder of the paper describes these two components, as well as a visualisation of the output of the decision tree in order to provide a rich summary of the courtship behaviour. A final classifier is used to separate markedly different courtship phenotypes based on the information extracted from the videos.

\section{Tracking}

The video data consists of (at least one set of) a pair of courting flies. In order to decide whether or not courtship is taking place, we need to extract the position and orientation of each fly in the video.

\begin{enumerate}
	\item create background model
	\item for each frame:
	\begin{enumerate}
		\item remove codec artefacts
		\item remove background model
		\item threshold
		\item mask
		\item find contours
		\item fit ellipses
		\item filter inappropriate ellipses
		\item choose the two ellipses closest to the previous ellipses
		\item choose orientation based on movement
	\end{enumerate}
\end{enumerate}

\section{Classification}

Description of the courtship classification component

\section{Visualisation}

Different visualisation options

\section{Classifying Courtship Dynamics}

Parmeterisation of the courtship index curves

\section{Performance}

Performance of the algorithm

\section{Conclusion}

Conclusion

\end{document}